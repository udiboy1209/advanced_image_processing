\documentclass{article}
\usepackage[a4paper, tmargin=1in, bmargin=1in]{geometry}
\usepackage[utf8]{inputenc}
\usepackage{graphicx}
\usepackage[justification=centering]{caption}

% \usepackage{parskip}
\usepackage{pdflscape}
\usepackage{listings}
\usepackage{hyperref}
\usepackage{caption}
\usepackage{subcaption}
\usepackage{float}
\usepackage{enumerate}
\usepackage{amsmath}

\setlength{\parindent}{0pt}

\title{CS 754 : Advanced Image ProcessingAssignment 2}
\author{Meet Udeshi - 14D070007\\
  Arka Sadhu - 140070011\\
}
\date{\today}

\newcommand{\lone}[1]{
  ||#1||_{l_1}
}
\newcommand{\ltwo}[1]{
  ||#1||_{l_2}
}
\newcommand{\linf}[1]{
  ||#1||_{l_\infty}
}
\newcommand{\htj}[1]{
  h_{T_{#1}}
}

\newcommand{\htc}[1]{
  h_{{#1}^c}
}

\newcommand{\hto}[1]{
  h_{#1}
}
\newcommand{\htzo}{
  h_{(T_0 \cup T_1)}
}

\newcommand{\htzoc}{
  h_{(T_0 \cup T_1)^c}
}

\begin{document}
\maketitle
\section*{Q1}
\subsection*{A1.1}
Need to show :
\begin{equation}\label{eq:1}
  \ltwo{h_{T_j}} \le s^{1/2} \linf{h_{T_j}}
\end{equation}
Equivalently we need to show :
$$\ltwo{A} \le s^{1/2} \linf{A}$$
where $A$ is a s-sparse vector.
Therefore
$$\ltwo{A} = \sqrt{\sum_{i}a_i^2} \le \sqrt{\sum_i max(a_i)^2} \le s^{1/2}max(a_i) = \linf{A}$$
The $s^{1/2}$ term comes from the fact that A is s-sparse matrix, and hence there will be at most s non-zero elements.

\subsection*{A1.2}
Need to show :
\begin{equation}\label{eq:2}
  s^{1/2}\linf{h_{T_j}} \le s^{-1/2}\lone{\htj{j-1}}
\end{equation}

for all $j \ge 2$\\\\
Equivalently we need to show :
$$s\linf{\htj{j}} \le \lone{\htj{j-1}}$$
From the definition of $T_j$ it follows for $j \ge 2$ that all elements of $\htj{j}$ will be less than the smallest non-zero element of
$\htj{j-1}$. Also both $\htj{j}$ and $\htj{j-1}$ are s-sparse matrix, hence it clearly follows that
$$s \linf{\htj{j}} = s*max(\htj{j}) \le \sum_i |\htj{j-1}| = \lone{\htj{j-1}} $$

\subsection*{A1.3}
Need to show :
\begin{equation}\label{eq:3}
  \sum_{j \ge 2} \ltwo{\htj{j}} \le s^{-1/2}(\lone{\htj{1}} + \lone{\htj{2}} + ...)
\end{equation}

This follows directly from \ref{eq:1} and \ref{eq:2}.
$$\ltwo{\htj{j}} \le s^{-1/2} \lone{\htj{j-1}}$$
for all $j \ge 2$
Now summing over all $j \ge 2$ we get
$$\sum_{j \ge 2} \ltwo{\htj{j}} \le s^{-1/2}(\lone{\htj{1}} + \lone{\htj{2}} + ...)$$

\subsection*{A1.4}
??
Need to show:
\begin{equation}
  \label{eq:4}
  s^{-1/2}(\lone{\htj{1}} + \lone{\htj{2}} + ...) \le s^{-1/2}\lone{\htc{T_0}}
\end{equation}

\subsection*{A1.5}
Need to show:
\begin{equation}
  \label{eq:5}
  \ltwo{\htc{(T_0 \cup T_1)}} = \ltwo{\sum_{j \ge 2}\htj{j}}
\end{equation}

We note that $$\htc{(T_0 \cup T_1)} = h - \hto{T_0} - \hto{T_1} = \hto{T_2} + \hto{T_3} + ... = \sum_{j \ge 2} \hto{T_j}$$

And hence \ref{eq:5} follows directly.

\subsection*{A1.6}
Need to show:
\begin{equation}
  \label{eq:6}
  \ltwo{\sum_{j \ge 2}{\htj{j}}} \le \sum_{j \ge 2} \ltwo{\htj{j}}
\end{equation}

This is simple extension of triangle inequality, which states that
$$|a+b| \le |a| + |b|$$
For n vectors it is simply
$$|a_1 + a_2 + a_3 +... +a_n| \le |a_1| + |a_2| + |a_3| + ... + |a_n| $$
And hence \ref{eq:6} follows directly

\subsection*{A1.7}
Need to show:
\begin{equation}
  \label{eq:7}
  \sum_{j \ge 2} \ltwo{\htj{j}} \le s^{-1/2}\lone{\htc{T_0}}
\end{equation}

This follows directly from \ref{eq:3} and \ref{eq:4}.
$$\sum_{j \ge 2} \ltwo{\htj{j}} \le s^{-1/2}(\lone{\htj{1}} + \lone{\htj{2}} + ...) \le s^{-1/2}\lone{\htc{T_0}}$$
\subsection*{A1.8}
???
Need to show:
\begin{equation}
  \label{eq:8}
  \lone{x} \ge \lone{x+h} \ge \lone{x_{T_0}} - \lone{\hto{T_0}} + \lone{\htc{T_0}} - \lone{x_{T_0^c}}
\end{equation}

\subsection*{A1.9}
Need to show:
\begin{equation}
  \label{eq:9}
  \ltwo{\htc{T_0}} \le \lone{\hto{T_0}} + 2\lone{x_{T_0^c}}
\end{equation}
We can rearrange \ref{eq:8} to have
$$\lone{\htc{T_0}} \le \lone{x} - \lone{x_{T_0}} + \lone{\hto{T_0}} + \lone{x_{T_0^c}}$$
We note that
$$\lone{x} - \lone{x_{T_0}} \le \lone{x - x_{T_0}} = \lone{x_{T_0^c}}$$
Therefore
$$\lone{\htc{T_0}} \le \lone{x_{T_0^c}} + \lone{\hto{T_0}} + \lone{x_{T_0^c}} = \lone{\hto{T_0}} + 2\lone{x_{T_0^c}}$$

\subsection*{A1.10}
Need to show:
\begin{equation}
  \label{eq:10}
  \ltwo{\htzoc} \le \ltwo{\hto{T_0}} + 2e_0, e_0 \equiv s^{-1/2}\ltwo{x - x_s}
\end{equation}

Combining \ref{eq:5} \ref{eq:6} and \ref{eq:7} we get
$$\ltwo{\htzoc} = \ltwo{\sum_{j \ge 2}\htj{j}} \le \sum_{j \ge 2} \ltwo{\htj{j}} \le s^{-1/2}\lone{\htc{T_0}}$$
From \ref{eq:9} we get
$$s^{-1/2}\lone{\htc{T_0}} \le s^{-1/2}\lone{\hto{T_0}} + 2s^{-1/2}\lone{x_{T_0^c}}$$
Also by definition
$$\lone{x_{T_0^c}} = \lone{x - x_s}$$
This implies
$$s^{-1/2}\lone{\htc{T_0}} \le s^{-1/2}\lone{\hto{T_0}} + 2s^{-1/2}\lone{x - x_s}$$
Therefore
$$s^{-1/2}\lone{\htc{T_0}} \le s^{-1/2}\lone{\hto{T_0}} + 2e_0, e_0 \equiv s^{-1/2}\ltwo{x - x_s}$$
Now we also note, for any s-sparse vector A
$$\lone{A} = \sum_i|a_i| = \sum_i |a_i|*1 \le \sqrt{s}\sqrt{\sum_i a_i^2} = s^{1/2}\ltwo{A}$$
and here we have used Cauchy Schwartz Inequality.
That is
$$s^{-1/2}\lone{A} \le \ltwo{A}$$
Thus it follows that 
$$s^{-1/2}\lone{\hto{T_0}} \le \ltwo{\hto{T_0}}$$
And \ref{eq:10} directly follows
$$\ltwo{\htzoc} \le \ltwo{\hto{T_0}} + 2e_0, e_0 \equiv s^{-1/2}\ltwo{x - x_s}$$

\subsection*{A1.14}
Need to show:
\begin{equation}
  \label{eq:14}
  \Phi \htzo = \Phi h - \sum_{j \ge 2}\Phi \htj{j}
\end{equation}
We know
$$\htzoc = h - \htzo =h - \hto{T_o} - \hto{T_1} = \sum_{j \ge 2}\htj{j}$$
Rearranging the equation
$$\htzo = h - \sum_{j \ge 2}\htj{j}$$
Multiplying $\phi$ on both sides
$$\Phi \htzo = \Phi h - \sum_{j \ge 2}\Phi \htj{j}$$

\subsection*{A1.15}
Need to show:
\begin{equation}
  \label{eq:15}
  |<\Phi\htzo,\Phi h>| \le \ltwo{\Phi\htzo}\ltwo{\Phi h}
\end{equation}

This is simple application of Cauchy Schwartz Inequality which states that given two vectors $a$ and $b$
$$<a,b> \le \ltwo{a}\ltwo{b}$$
And therefore \ref{eq:15} directly follows from this.

\subsection*{A1.16}
Need to show:


\end{document}